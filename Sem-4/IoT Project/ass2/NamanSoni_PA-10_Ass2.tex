\documentclass{article}
\usepackage[utf8]{inputenc}
\usepackage[margin=1in]{geometry}
\usepackage{listings}
\usepackage{xcolor}
\usepackage{booktabs}
\usepackage{graphicx}

\definecolor{codegreen}{rgb}{0,0.6,0}
\definecolor{codegray}{rgb}{0.5,0.5,0.5}
\definecolor{codepurple}{rgb}{0.58,0,0.82}
\definecolor{backcolour}{rgb}{0.95,0.95,0.92}

\lstdefinestyle{mystyle}
{
	backgroundcolor=\color{backcolour},   
	commentstyle=\color{codegreen},
	keywordstyle=\color{magenta},
	numberstyle=\tiny\color{codegray},
	stringstyle=\color{codepurple},
	basicstyle=\ttfamily\footnotesize,
	breakatwhitespace=false,         
	breaklines=true,                 
	captionpos=b,                    
	keepspaces=true,                 
	numbers=left,                    
	numbersep=5pt,                  
	showspaces=false,                
	showstringspaces=false,
	showtabs=false,                  
	tabsize=2
}

\lstset{style=mystyle}
\begin{document}
\begin{titlepage} % Suppresses displaying the page number on the title page and the subsequent page counts as page 1
	
	\raggedleft\rule{1pt}{\textheight} % Vertical line
	\hspace{0.05\textwidth} % Whitespace between the vertical line and title page text
	\parbox[b]{0.75\textwidth}
	{ % Paragraph box for holding the title page text, adjust the width to move the title page left or right on the page
		
		{\Huge\bfseries MIT World Peace University \\[0.5\baselineskip] \ Internet of Things}\\[2\baselineskip] % Title
		{\large\textit{Assignment 2}}\\[4\baselineskip] % Subtitle or further description
		{\Large\textsc{Naman Soni Roll No. 10}} % Author name, lower case for consistent small caps
		
		\vspace{0.5\textheight} % Whitespace between the title block and the publisher
	}
	
\end{titlepage}
\tableofcontents
\pagebreak
\section{\textbf{Aim}}
To interface following Sensors such as Temperature or Ultrasonic or IR or any other sensor with Arduino Uno and display the output on the Serial Monitor.
\section{\textbf{Objectives}}
\begin{itemize}
	\item  To interface Temperature Sensor with Arduino Uno and display the output on the Serial Mon itor.
	\item To learn how to use Arduino Uno.
	\item To learn about Actuators and Sensors.
\end{itemize}
\section{\textbf{Components and Equipment}}
\begin{itemize}
	\item  U1:1   
	\item Arduino Uno R3
	\item U5:1
	\item Temperature Sensor [TMP36] PIEZO1: 1
	\item Piezo
	\item D1: 1
	\item Red LED
	\item MFAN: 1
	\item DC Motor
	\item R1: 1
	\item 1 kΩ Resistor
\end{itemize}
\section{\textbf{Theory}}
In this assignment, we will be interfacing a temperature sensor with Arduino Uno and displaying
the output on the serial monitor. The temperature sensor used in this project is an analog sensor, which measures temperature by outputting a voltage proportional to the temperature. The TMP36 temperature sensor is a popular choice due to its accuracy and low cost.
The Arduino Uno is a microcontroller board that is commonly used for prototyping and educa
tional purposes. It has 14 digital input/output pins and 6 analog input pins, and is powered by a 5V supply. The board is programmed using the Arduino programming language, which is a simplified version of C++.
To interface the temperature sensor with the Arduino, we connect the output pin of the sensor to one of the analog input pins on the Arduino, and connect the ground and power pins to the appro priate pins on the Arduino. We then read the analog voltage from the sensor using the analogRead() function in the Arduino code.
The output of the temperature sensor is displayed on the serial monitor, which is a useful tool
for debugging and monitoring the output of the Arduino. The serial monitor allows us to view the output in real-time and make any necessary adjustments to the code or hardware.
In the context of the Internet of Things (IoT), this project can be extended to include wireless
communication, allowing the temperature readings to be monitored and analyzed remotely. For example, the Arduino could be connected to a Wi-Fi module or a cellular module, and the temperature readings could be sent to a cloud-based platform for analysis and visualization. This could be useful in applications such as environmental monitoring or industrial automation
\section{\textbf{Platform}}
\textit{\underline{Operating System:}} Mac OS 64-bit\\
\textit{\underline{IDEs or Text Editors Used:}} Arduino IDE, and Thonny on Pi\\ 
\textit{\underline{Compilers:}} g++ and gcc on linux for C++, Python 3.10 on Pi
\section{\textbf{Diagrams}}
\begin{center}
	\includegraphics[scale=0.4]{}
\end{center}
\begin{center}
	\textbf{}
\end{center}
\vspace{10pt}
\begin{center}
	\includegraphics[scale=0.4]{}
\end{center}
\begin{center}
	\textbf{}
\end{center}
\vspace{10pt}
\begin{center}
	\includegraphics[scale=0.4]{}
\end{center}
\begin{center}
	\textbf{}
\end{center}

\section{\textbf{Conclusion}}

\section{\textbf{FAQ's}}
\subsection{\textit{Arduino Uno R3 (Code: U1)}}
\end{document}