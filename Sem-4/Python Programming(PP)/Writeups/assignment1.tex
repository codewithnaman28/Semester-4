\documentclass{article}
\usepackage[utf8]{inputenc}
\usepackage[margin=1in]{geometry}
\usepackage{listings}
\usepackage{xcolor}
\usepackage{booktabs}

\definecolor{codegreen}{rgb}{0,0.6,0}
\definecolor{codegray}{rgb}{0.5,0.5,0.5}
\definecolor{codepurple}{rgb}{0.58,0,0.82}
\definecolor{backcolour}{rgb}{0.95,0.95,0.92}

\lstdefinestyle{mystyle}
{
    backgroundcolor=\color{backcolour},   
    commentstyle=\color{codegreen},
    keywordstyle=\color{magenta},
    numberstyle=\tiny\color{codegray},
    stringstyle=\color{codepurple},
    basicstyle=\ttfamily\footnotesize,
    breakatwhitespace=false,         
    breaklines=true,                 
    captionpos=b,                    
    keepspaces=true,                 
    numbers=left,                    
    numbersep=5pt,                  
    showspaces=false,                
    showstringspaces=false,
    showtabs=false,                  
    tabsize=2
}

\lstset{style=mystyle}
\begin{document}
\begin{titlepage} % Suppresses displaying the page number on the title page and the subsequent page counts as page 1

	\raggedleft\rule{1pt}{\textheight} % Vertical line
	\hspace{0.05\textwidth} % Whitespace between the vertical line and title page text
	\parbox[b]{0.75\textwidth}
    { % Paragraph box for holding the title page text, adjust the width to move the title page left or right on the page
		
		{\Huge\bfseries MIT World Peace University \\[0.5\baselineskip] \ Python Programming}\\[2\baselineskip] % Title
		{\large\textit{Assignment 1}}\\[4\baselineskip] % Subtitle or further description
		{\Large\textsc{Naman Soni Roll No. 10}} % Author name, lower case for consistent small caps
		
		\vspace{0.5\textheight} % Whitespace between the title block and the publisher
	}

\end{titlepage}
\tableofcontents
\pagebreak
\section{\textbf{Problem Statement}}
Introduction to basic Python Commands.
\section{\textbf{Aim}}
To learn the basics of the python programming language and understand fundamental syntax and semantics of Python Programming.
\section{\textbf{Objectives}}
\begin{enumerate}
    \item To learn the basics of python programming language.
    \item To learn the variable declaration,user input and output of the programming language.
\end{enumerate}
\section{\textbf{Theory}}
\subsection{\textbf{\textit{Introduction to Python}}}
Python is a high-level, interpreted programming language that was first released in 1991. It is known for its simple and easy-to-read syntax, making it a popular choice for beginners and experienced developers alike. Python can be used for a wide range of tasks, including web development, scientific computing, data analysis, artificial intelligence, and more. It is an open-source language with a large and supportive community, and has a vast library of modules and tools available to users.
\subsection{\textbf{\textit{Basic Commands in Python}}}
\begin{enumerate}
    \item Print: display a message on the screen. For example, print (``Hello World '').
    \item Variable assignment: assign a value to a variable. For example, x = 5.
    \item Arithmetic operations: perform mathematical operations such as addition, subtraction, multiplication, and division. For example, x + 5, x - 5, x * 5, and x / 5
    \item Conditional statements: make decisions based on conditions. For example, if x > 5: print (``x is greater than 5'').
    \item Loops: repeat a block of code multiple times. For example, for i in range (5): print (i).
    \item Importing modules: use pre-existing code from other sources. For example, import math to use mathematical functions.
\end{enumerate}
\subsection{\textbf{\textit{Standard Data Types}}}
The data stored in memory can be of many types. Python has various standard data types
that are used to define the operations possible on them and the storage method for each of
them.\\
Python has 5 standard data types:
\begin{itemize}
    \item Numbers
    \item String
    \item List 
    \item Tuple 
    \item Dictionary
\end{itemize}
\section{\textbf{Platform}}
Python\\
Mac OS 64-bit\\
Visual Studio Code
\section{\textbf{Code Input/Output}}
\begin{lstlisting}[language=python,caption=print function input]
    print("Hello world!")
\end{lstlisting}
\begin{lstlisting}
    Hello world!
\end{lstlisting}
\begin{lstlisting}[language=python,caption=declaring variable input]
# declare variable
a = 3
print(a)
print(type(a))
b = 3.5
print(type(b))
print(b)
c = "Hello"
print(c)
print(type(c))
\end{lstlisting}
\begin{lstlisting}
3
<class 'int'>
<class 'float'>
3.5
Hello
<class 'str'>
\end{lstlisting}
\begin{lstlisting}[language=python,caption=performing arithmetic operation input]
a = 5
b = 6
print(a + b)
\end{lstlisting}
\begin{lstlisting}
11
\end{lstlisting}
\begin{lstlisting}[language=python]
a = input("Enter First Name:")
b = input("Enter Last Name:")
print(a + b)
\end{lstlisting}
\begin{lstlisting}
naman soni 
\end{lstlisting}
\begin{lstlisting}[language=python]
x = int(input("Enter the first number:"))
y = int(input("Enter the second number:"))
z = x + y
print("Sum is:", z)
\end{lstlisting}
\begin{lstlisting}
Sum is: 66
\end{lstlisting}
\begin{lstlisting}[language=python]
x = float(input("Enter the first number:"))
y = float(input("Enter the second number:"))
z = x + y
print("Sum is:", z)
\end{lstlisting}
\begin{lstlisting}
Sum is: 110.6943    
\end{lstlisting}
\begin{lstlisting}[language=python]
a = 45
b = 65
if (a == b):
    print("equal")
else:
    print("not equal")
\end{lstlisting}
\begin{lstlisting}
not equal
\end{lstlisting}
\begin{lstlisting}[language=python]
my_list = [12, 13, 56, 43]
print(my_list)
\end{lstlisting}
\begin{lstlisting}
[12, 13, 56, 43]
\end{lstlisting}

\end{document}
