\documentclass{article}
\usepackage[utf8]{inputenc}
\usepackage[margin=1in]{geometry}
\usepackage{listings}
\usepackage{xcolor}
\usepackage{booktabs}

\definecolor{codegreen}{rgb}{0,0.6,0}
\definecolor{codegray}{rgb}{0.5,0.5,0.5}
\definecolor{codepurple}{rgb}{0.58,0,0.82}
\definecolor{backcolour}{rgb}{0.95,0.95,0.92}

\lstdefinestyle{mystyle}
{
    backgroundcolor=\color{backcolour},   
    commentstyle=\color{codegreen},
    keywordstyle=\color{magenta},
    numberstyle=\tiny\color{codegray},
    stringstyle=\color{codepurple},
    basicstyle=\ttfamily\footnotesize,
    breakatwhitespace=false,         
    breaklines=true,                 
    captionpos=b,                    
    keepspaces=true,                 
    numbers=left,                    
    numbersep=5pt,                  
    showspaces=false,                
    showstringspaces=false,
    showtabs=false,                  
    tabsize=2
}

\begin{document}
\lstset{style=mystyle}
\begin{titlepage} % Suppresses displaying the page number on the title page and the subsequent page counts as page 1

	\raggedleft\rule{1pt}{\textheight} % Vertical line
	\hspace{0.05\textwidth} % Whitespace between the vertical line and title page text
	\parbox[b]{0.75\textwidth}
    { % Paragraph box for holding the title page text, adjust the width to move the title page left or right on the page
		
		{\Huge\bfseries MIT World Peace University \\[0.5\baselineskip] \ \textit{Analysis of Algorithms}}\\[2\baselineskip] % Title
		{\large\textit{Unit 1}}\\[4\baselineskip] % Subtitle or further description
		{\Large\textsc{Naman Soni Roll No. 10}} % Author name, lower case for consistent small caps
		
		\vspace{0.5\textheight} % Whitespace between the title block and the publisher
	}

\end{titlepage}
\tableofcontents
\pagebreak
\section{Divide and Conquer}
\subsection{Control Abstraction}
\begin{lstlisting}
    DANDC (P)
    {
        if SMALL (P) then return S (p);
        else
        {
            divide p into smaller instances p1, p2,...Pk, k>=1;
            apply DANDC to each of these sub problems;
            return (COMBINE (DANDC (p1), DANDC (P2),...,DANDC (pk)));
        }
    }
\end{lstlisting}
\subsection{Time Complexity of the general algorithm}
\begin{itemize}
    \item A recurrence is an equation or inequality that describes a function in terms of its value on smaller inputs.
    \item Special techniques are required to analyze the space and time required.
    \item $T(n) = {\frac{aT(\frac{n}{b}+1)(n)+c(n)}{O(1)}}$
    \item Time Complexity (recurrence relation):
    (\begin{itemize}
        \item where D(n): time for splitting 
        \item C(n): time for conquer
        \item c: a constant
    \end{itemize})
\end{itemize}
\subsection{Methods for Solving recurrences}
\begin{enumerate}
    \item Substitution method: This method involves guessing a solution and then proving that it is correct.
    \item Recurrence tree method: This method involves constructing a tree diagram that represents the recursive calls and their relationship to each other.
    \item Master theorem: This is a general theorem that provides a method for solving recurrences of a specific form.
\end{enumerate}
\subsection{Math you need to Review}
Properties of Logarithms:
\begin{itemize}
    \item $\log_b (xy) = \log_b(x) + \log_b(y)$
    \item $\log_b(\frac{x}{y}) = \log_b(x) - \log_b(y)$
    \item $\log_b xa = a\log_b x$
    \item $\log_b a = \frac{\log_x a}{\log_b b}$
\end{itemize}
Properties of exponentials:
\begin{itemize}
    \item $a^{(b+c)} = a^b a^c$
    \item $a^{bc} = (a^b) ^c$
    \item $\frac{a^b}{a^c} = a^{(b-c)}$
    \item $b = a^{\log_a b}$
    \item $b^c = a^{c^*\log_a b}$
\end{itemize}
\section{Divide-and-Conquer Examples}
\begin{itemize}
	\item Sorting: mergesort and Quick Sort
	\item Binary tree traversals
	\item Binary search 
	\item Multiplication of Large integers 
	\item Matrix Multiplicatoion: Strassen's algorithm 
	\item Closet-pair and convex-hull algorithms
\end{itemize}
\section{Proof Techniques}
\begin{itemize}
	\item Proof is a kind of demonstration to convince that the given mathematical statement is true.
	\item The statement which is to be proved is called theorem. Once a particulare theorem is proved then it can be used to prove further statements.
	\item The theorem is also called Lemma.
	\item The proof can be a deductive proof or inductive proof.
	\item The deductive proof consist of sequence of statements given with logical reasoning.
	\item The inductive proof is a recursive kind of proof which consists of sequence or parameterized statements that use the statement itself or the statement with lower values of its parameters/.
\end{itemize}
\subsection{Proof by contradiction}
\begin{itemize}
	\item In this type of proof, for the statement of this form is A and the B. 
	\item Prove by contradiction. There exist two irrational numbers x and y such that $x^y$ is rational.
	\begin{itemize}
		\item An irrational number is any number taht cannot be expressed as a/b where a and b are integers and value b is non zero. To prove that $x^y$ is rational when x and y are rational numbers.
		
	\end{itemize}
\end{itemize}
\subsection{Proof by Mathematical Induction}
\begin{itemize}
	\item Prove $ 1+ 2+3...+n = n(n+1)/2$
	\begin{itemize}
		\item 1) Basis of incution
		\item Assume, n = 1 then 
		\item LHS = n = 1
		\item RHS =$ n(n+1)/2 = 1 (1+1)/2 = 2/2 =1$
		\item 2) Induction hypothesis
		\item Now we will assume n = K and will obtain the result for it. The equation then becomes,
		\item $ 1 + 2 + 3 + ... + K = K(K=1)/2 $
		\item 3) Inductive step
		\item Now we assume that equation is true for n = K and we willl then check if it is also true for n = K + 1 or not.
		\item Consider the equqtion assuming n = K + 1
		\item LHS = 1 + 2 + 3 + ... + K + K  + 1
		\item = $ K ( K + 1)/2 + K + 1$	
		\item = $ K ( K + 1) + 2 (K + 1)/2$
		\item = $ ( K + 1) ( K + 2)/2$
	\end{itemize}
\end{itemize}
\subsection{Direct Proof}
\begin{itemize}
	\item In direct Proof, the intended proof can be proved by basic principle or axiom.
	\item Example:- Prove that the negative of any even integer is even.
	\item Solution: to prove this, let n be any positive even number. Hence we can write n as 
	\item $n = 2m$		where m can be any number
	\item if we multiply both side by -1, we get 
	\item $-n = -2m$
	\item $-n = 2 (-m)$
	\item Multiplying any number by 2 makes it an even number.
	\item Hence, -n is even.
	\item Thus proves that the negative of any even integer is even.
\end{itemize}
\subsection{Proof by Counter-Example}
\begin{itemize}
	\item This is a technique of proof in which a->b is true if ~a->~b
	\item If negative statement of given statemetn is true then the given statement becomes automatically true.
	\item Example: prove contraposition that $ x + 8$ is odd
	\item Solution:
		\begin{itemize}
			\item Step 1: we assume that x is not odd
			\item Step 2: that means x is even. By definition of even numbers 2 * any number = even number.
			\item $ x = 2 * m$ 		where m can be any number
			\item we can write $ x+ 8$ as $ 2*m+8=2(m+4)= even number$
			\item thus, $ x + 8$ is even 
		\end{itemize}
\end{itemize}
\section{Brute Force}
\begin{itemize}
	\item Burete force is a straightforward approach to solve a problem based on the problem's statement and definitions of the concepts involved.
	\item It is condidered as one of the easiest approach to apply and is useful for solving small-size instances of a problem.
\end{itemize}
\subsection{Brute Force Search and Sort}
\begin{itemize}
	\item Sequential search in an unordered array and simple sorts- selection sort, bubble sort are brute force algorithm.
	\item Sequential search: the algorithm simply compared successive elements of a given list with a given search key until either a match is found or the list is exhausted without finding a match.
\end{itemize}
\end{document}