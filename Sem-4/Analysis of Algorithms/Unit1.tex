\documentclass{article}
\usepackage[utf8]{inputenc}
\usepackage[margin=1in]{geometry}
\usepackage{listings}
\usepackage{xcolor}
\usepackage{booktabs}

\definecolor{codegreen}{rgb}{0,0.6,0}
\definecolor{codegray}{rgb}{0.5,0.5,0.5}
\definecolor{codepurple}{rgb}{0.58,0,0.82}
\definecolor{backcolour}{rgb}{0.95,0.95,0.92}

\lstdefinestyle{mystyle}
{
    backgroundcolor=\color{backcolour},   
    commentstyle=\color{codegreen},
    keywordstyle=\color{magenta},
    numberstyle=\tiny\color{codegray},
    stringstyle=\color{codepurple},
    basicstyle=\ttfamily\footnotesize,
    breakatwhitespace=false,         
    breaklines=true,                 
    captionpos=b,                    
    keepspaces=true,                 
    numbers=left,                    
    numbersep=5pt,                  
    showspaces=false,                
    showstringspaces=false,
    showtabs=false,                  
    tabsize=2
}

\begin{document}
\lstset{style=mystyle}
\begin{titlepage} % Suppresses displaying the page number on the title page and the subsequent page counts as page 1

	\raggedleft\rule{1pt}{\textheight} % Vertical line
	\hspace{0.05\textwidth} % Whitespace between the vertical line and title page text
	\parbox[b]{0.75\textwidth}
    { % Paragraph box for holding the title page text, adjust the width to move the title page left or right on the page
		
		{\Huge\bfseries MIT World Peace University \\[0.5\baselineskip] \ \textit{Analysis of Algorithms}}\\[2\baselineskip] % Title
		{\large\textit{Unit 1}}\\[4\baselineskip] % Subtitle or further description
		{\Large\textsc{Naman Soni Roll No. 10}} % Author name, lower case for consistent small caps
		
		\vspace{0.5\textheight} % Whitespace between the title block and the publisher
	}

\end{titlepage}
\tableofcontents
\pagebreak
\section{Divide and Conquer}
\subsection{Control Abstraction}
\begin{lstlisting}
    DANDC (P)
    {
        if SMALL (P) then return S (p);
        else
        {
            divide p into smaller instances p1, p2,...Pk, k>=1;
            apply DANDC to each of these sub problems;
            return (COMBINE (DANDC (p1), DANDC (P2),...,DANDC (pk)));
        }
    }
\end{lstlisting}
\subsection{Time Complexity of the general algorithm}
\begin{itemize}
    \item A recurrence is an equation or inequality that describes a function in terms of its value on smaller inputs.
    \item Special techniques are required to analyze the space and time required.
    \item $T(n) = {\frac{aT(\frac{n}{b}+1)(n)+c(n)}{O(1)}}$
    \item Time Complexity (recurrence relation):
    (\begin{itemize}
        \item where D(n): time for splitting 
        \item C(n): time for conquer
        \item c: a constant
    \end{itemize})
\end{itemize}
\subsection{Methods for Solving recurrences}
\begin{enumerate}
    \item Substitution method: This method involves guessing a solution and then proving that it is correct.
    \item Recurrence tree method: This method involves constructing a tree diagram that represents the recursive calls and their relationship to each other.
    \item Master theorem: This is a general theorem that provides a method for solving recurrences of a specific form.
\end{enumerate}
\subsection{Math you need to Review}
Properties of Logarithms:
\begin{itemize}
    \item $\log_b (xy) = \log_b(x) + \log_b(y)$
    \item $\log_b(\frac{x}{y}) = \log_b(x) - \log_b(y)$
    \item $\log_b xa = a\log_b x$
    \item $\log_b a = \frac{\log_x a}{\log_b b}$
\end{itemize}
Properties of exponentials:
\begin{itemize}
    \item $a^{(b+c)} = a^b a^c$
    \item $a^{bc} = (a^b) ^c$
    \item $\frac{a^b}{a^c} = a^{(b-c)}$
    \item $b = a^{\log_a b}$
    \item $b^c = a^{c^*\log_a b}$
\end{itemize}

\end{document}