\documentclass{article}
\usepackage[utf8]{inputenc}
\usepackage[margin=1in]{geometry}
\usepackage{listings}
\usepackage{xcolor}
\usepackage{booktabs}
\usepackage{graphicx}

\definecolor{codegreen}{rgb}{0,0.6,0}
\definecolor{codegray}{rgb}{0.5,0.5,0.5}
\definecolor{codepurple}{rgb}{0.58,0,0.82}
\definecolor{backcolour}{rgb}{0.95,0.95,0.92}

\lstdefinestyle{mystyle}
{
	backgroundcolor=\color{backcolour},   
	commentstyle=\color{codegreen},
	keywordstyle=\color{magenta},
	numberstyle=\tiny\color{codegray},
	stringstyle=\color{codepurple},
	basicstyle=\ttfamily\footnotesize,
	breakatwhitespace=false,         
	breaklines=true,                 
	captionpos=b,                    
	keepspaces=true,                 
	numbers=left,                    
	numbersep=5pt,                  
	showspaces=false,                
	showstringspaces=false,
	showtabs=false,                  
	tabsize=2
}

\lstset{style=mystyle}
\begin{document}
\begin{titlepage} % Suppresses displaying the page number on the title page and the subsequent page counts as page 1
		
		\raggedleft\rule{1pt}{\textheight} % Vertical line
		\hspace{0.05\textwidth} % Whitespace between the vertical line and title page text
		\parbox[b]{0.75\textwidth}
		{ % Paragraph box for holding the title page text, adjust the width to move the title page left or right on the page
			
			{\Huge\bfseries MIT World Peace University \\[0.5\baselineskip] \ Data Base Management System}\\[2\baselineskip] % Title
			{\large\textit{Assignment 2}}\\[4\baselineskip] % Subtitle or further description
			{\Large\textsc{Naman Soni Roll No. 10}} % Author name, lower case for consistent small caps
			
			\vspace{0.5\textheight} % Whitespace between the title block and the publisher
		}
		
\end{titlepage}
\tableofcontents
\pagebreak
\section{\textbf{Aim}}
Design and Develop SQL DDL statements for different system.
\section{\textbf{Objective}}
To study DDL, DCL commands.
\section{\textbf{Theory}}
DDL stands for Data Definition Language and refers to a set of SQL commands used to create, modify, and delete database objects such as tables, views, indexes, and constraints. DDL commands include:
\begin{itemize}
    \item \textbf{CREATE:} used to create new database objects such as tables, views, and indexes.
    \item \textbf{ALTER:} used to modify the structure of existing database objects such as tables and views.
    \item \textbf{DROP:} used to delete database objects such as tables, views, and indexes.
    \item \textbf{TRUNCATE:} used to remove all data from a table, but keep its structure intact.
    \item \textbf{RENAME:} used to change the name of an existing database object.
\end{itemize}

On the other hand, DCL stands for Data Control Language and refers to a set of SQL commands used to control access to database objects. DCL commands include:
\begin{itemize}
    \item \textbf{GRANT:} used to provide specific privileges to a user or role to perform certain operations on database objects.
    \item \textbf{REVOKE:} used to revoke specific privileges from a user or role.
    \item \textbf{DENY:} used to prevent a user or role from performing specific operations on database objects.
\end{itemize}
\section{\textbf{Input}}
\section{\textbf{Output}}
\section{\textbf{Conclusion}}
Thus, we have learned DDL and DCL commands thoroughly.
\section{\textbf{FAQ's}}
\subsection{\textit{How to drop a column from a table?}}
To drop a column from a table, you can use the ALTER TABLE command with the DROP COLUMN clause. Here's the basic syntax:
\begin{lstlisting}[language=SQL]
    ALTER TABLE table_name
    DROP COLUMN column_name;
\end{lstlisting}
Here, table-name is the name of the table from which you want to drop the column, and column-name is the name of the column that you want to drop.

For example, if you have a table called ``employees'' with a column called ``phone-number'' that you want to drop, you can use the following SQL command:
\begin{lstlisting}[language=SQL]
    ALTER TABLE employees
    DROP COLUMN phone-number;
\end{lstlisting}
After executing this command, the ``phone-number'' column will be removed from the ``employees'' table. Note that this action cannot be undone, so be sure to make a backup of your database before making any changes to it.
\subsection{\textit{How to add a primary key in an already existing table?}}
You can add a primary key to an already existing table using the ALTER TABLE command in SQL. Here's the basic syntax:
\begin{lstlisting}[language=SQL]
    ALTER TABLE table_name
    ADD CONSTRAINT constraint_name PRIMARY KEY (column_name);
\end{lstlisting}
Here, table-name is the name of the table to which you want to add a primary key, constraint-name is the name of the constraint you want to give to the primary key, and column-name is the name of the column or columns that you want to include in the primary key.

For example, suppose you have a table called students with ``columns id'', ``name'', and ``email''. If you want to add a primary key constraint to the ``id'' column, you can use the following SQL command:
\begin{lstlisting}
    ALTER TABLE students
    ADD CONSTRAINT pk_students_id PRIMARY KEY (id);
\end{lstlisting}
After executing this command, the ``id'' column will be set as the primary key for the ``students'' table. Note that if there are any duplicate values in the ``id'' column, this command will fail, as primary keys must be unique for each row in the table.
\subsection{\textit{How to create a new user in MySQL?}}
To create a new user in MySQL, you can use the CREATE USER statement followed by the GRANT statement to grant privileges to the user. Here's the basic syntax:
\begin{lstlisting}[language=SQL]
    CREATE USER 'username'@'localhost' IDENTIFIED BY 'password';
    GRANT ALL PRIVILEGES ON *.* TO 'username'@'localhost';
    FLUSH PRIVILEGES;
\end{lstlisting}
Here, username is the name of the new user that you want to create, and password is the password that you want to assign to the user.

The GRANT statement grants all privileges to the new user on all databases and tables by using the wildcard symbol ``.''. You can modify this statement to grant specific privileges to the user on specific databases and tables.

The FLUSH PRIVILEGES statement is used to reload the privileges table in MySQL so that the new user can immediately access the database.

For example, if you want to create a new user called ``johndoe'' with the password ``password123'', you can use the following SQL commands:
\begin{lstlisting}[language=SQL]
    CREATE USER 'johndoe'@'localhost' IDENTIFIED BY 'password123';
    GRANT ALL PRIVILEGES ON *.* TO 'johndoe'@'localhost';
    FLUSH PRIVILEGES;
\end{lstlisting}
After executing these commands, the new user "johndoe" will be created in MySQL and will have all privileges on all databases and tables.
\end{document}