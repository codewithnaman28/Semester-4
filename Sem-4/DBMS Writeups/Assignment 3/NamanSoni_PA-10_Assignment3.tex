\documentclass{article}
\usepackage[utf8]{inputenc}
\usepackage[margin=1in]{geometry}
\usepackage{listings}
\usepackage{xcolor}
\usepackage{booktabs}

\definecolor{codegreen}{rgb}{0,0.6,0}
\definecolor{codegray}{rgb}{0.5,0.5,0.5}
\definecolor{codepurple}{rgb}{0.58,0,0.82}
\definecolor{backcolour}{rgb}{0.95,0.95,0.92}

\lstdefinestyle{mystyle}
{
	backgroundcolor=\color{backcolour},   
	commentstyle=\color{codegreen},
	keywordstyle=\color{magenta},
	numberstyle=\tiny\color{codegray},
	stringstyle=\color{codepurple},
	basicstyle=\ttfamily\footnotesize,
	breakatwhitespace=false,         
	breaklines=true,                 
	captionpos=b,                    
	keepspaces=true,                 
	numbers=left,                    
	numbersep=5pt,                  
	showspaces=false,                
	showstringspaces=false,
	showtabs=false,                  
	tabsize=2
}

\lstset{style=mystyle}
\begin{document}
	\begin{titlepage} % Suppresses displaying the page number on the title page and the subsequent page counts as page 1
		
		\raggedleft\rule{1pt}{\textheight} % Vertical line
		\hspace{0.05\textwidth} % Whitespace between the vertical line and title page text
		\parbox[b]{0.75\textwidth}
		{ % Paragraph box for holding the title page text, adjust the width to move the title page left or right on the page
			
			{\Huge\bfseries MIT World Peace University \\[0.5\baselineskip] \ Data Base Management System}\\[2\baselineskip] % Title
			{\large\textit{Assignment 3}}\\[4\baselineskip] % Subtitle or further description
			{\Large\textsc{Naman Soni Roll No. 10}} % Author name, lower case for consistent small caps
			
			\vspace{0.5\textheight} % Whitespace between the title block and the publisher
		}
		
	\end{titlepage}
	\tableofcontents
	\pagebreak
	\section{\textbf{Aim}} 
	Write suitable DML and select command to manipulate and retrieve requested data from tables. 
	
	\section{\textbf{Objective}} 
	\begin{enumerate} 
		\item DML (Insert, Update, Delete) commands, 
		\item SQL Select- Logical, IN, Negation, NULL, Comparison Operators. 
		\item Where Clause, Between AND, Exists, ALL, LIKE 
	\end{enumerate} 
	
	\section{\textbf{Theory}} 
	
	\subsection{\textbf{ SQL Data Manipulation Language (DML)}} 
	\subsubsection{\textbf{what is DML?}} 
	DML stands for Data Manipulation Language. It is used to manipulate the data in the database. It is used to insert, update, delete, and retrieve data from the database. 
	\subsubsection{\textbf{DML Commands}} 
	The following are the DML commands:  
	\begin{itemize} 
		\item \textbf{INSERT} - inserts new records in a table. 
		\item \textbf{UPDATE} - updates existing records in a table. 
		\item \textbf{DELETE} - deletes existing records from a table. 
		\item \textbf{SELECT} - retrieves records from one or more tables. 
	\end{itemize} 
	
	\subsection{\textbf{DML Commands Syntax and Examples}} 
	\begin{itemize} 
		\item \textbf{SELECT} - retrieves records from one or more tables. 
		\begin{lstlisting}[language=SQL] 
			SELECT column_name 1 , column_name 2 , ... FROM table_name; 
		\end{lstlisting} 
		\item \textbf{INSERT} - inserts new records in a table. 
		\begin{lstlisting}[language=SQL] 
			INSERT INTO table_name (column_name 1 , column_name 2 , ...) VALUES (value 1 , value 2 , ...); 
		\end{lstlisting} 
		\item \textbf{UPDATE} - updates existing records in a table. 
		\begin{lstlisting}[language=SQL] 
			UPDATE table_name SET column_name = value WHERE condition; 
		\end{lstlisting} 
		\item \textbf{DELETE} - deletes existing records from a table. 
		\begin{lstlisting}[language=SQL] 
			DELETE FROM table_name WHERE condition; 
		\end{lstlisting} 
	\end{itemize} 
	
	\subsection{\textbf{SELECT query}} 
	\subsubsection{\textbf{what is SELECT query?}} 
	The SELECT statement is used to select data from a database. The data returned is stored in a result table, called the result-set. 
	\subsubsection{\textbf{SELECT query Syntax}} 
	\begin{lstlisting}[language=SQL] 
		SELECT column_name 1 , column_name 2 , ... FROM table_name WHERE coloumn_name = value; 
	\end{lstlisting} 
	
	\subsubsection{\textbf{SELECT Operators}} 
	The following are the operators used in SELECT query: 
	\begin{enumerate} 
		\item \textbf{AND:} The AND operator displays a record if both the first condition AND the second condition are true. 
		\item \textbf{OR:} The OR operator displays a record if either the first condition OR the second condition is true. 
		\item \textbf{NOT:} The NOT operator displays a record if the condition is NOT true. 
		\item \textbf{BETWEEN:} The BETWEEN operator selects values within a given range. The values can be numbers, text, or dates. 
		\item \textbf{IN:} The IN operator allows you to specify multiple values in a WHERE clause. 
		\item \textbf{LIKE:} The LIKE operator is used in a WHERE clause to search for a specified pattern in a column. 
	\end{enumerate} 
	
	\subsubsection{\textbf{SELECT query Examples}} 
	
	\begin{lstlisting}[language=SQL] 
		SELECT * FROM table_name WHERE coloumn_name = value; 
		SELECT * FROM table_name WHERE coloumn_name = value AND coloumn_name = value; 
		SELECT * FROM table_name WHERE coloumn_name = value OR coloumn_name = value; 
		SELECT * FROM table_name WHERE NOT coloumn_name = value; 
		SELECT * FROM table_name WHERE coloumn_name BETWEEN value AND value; 
		SELECT * FROM table_name WHERE coloumn_name IN (value, value, value); 
		SELECT * FROM table_name WHERE coloumn_name LIKE 'value'; 
	\end{lstlisting} 
	
	\subsection{\textbf{SQL Operators}} 
	\subsubsection{\textbf{what is SQL Operators?}} 
	An operator is a symbol that tells the compiler to perform specific mathematical or logical manipulations. 
	
	\subsubsection{\textbf{SQL Operators}} 
	The following are the operators used in SQL: 
	\begin{enumerate} 
		\item \textbf{Arithmetic Operators:} Arithmetic operators are used to perform mathematical operations like addition, subtraction, multiplication, and division. 
		\item \textbf{Comparison Operators:} Comparison operators are used to compare values. It returns either true or false according to the condition. 
		\item \textbf{Logical Operators:} Logical operators are used to determine the logic between variables or values. 
		\item \textbf{Misc Operators:} Misc operators are used to perform miscellaneous operations. 
	\end{enumerate} 
	
	\subsubsection{\textbf{Arithmetic Operators}} 
	The following are the Arithmetic Operators used in SQL: 
	\begin{enumerate} 
		\item \textbf{+} - Addition 
		\item \textbf{-} - Subtraction 
		\item \textbf{*} - Multiplication 
		\item \textbf{/} - Division 
		\item \textbf{\%} - Modulus 
	\end{enumerate} 
	
	\subsubsection{\textbf{Comparison Operators}} 
	The following are the Comparison Operators used in SQL: 
	\begin{enumerate} 
		\item \textbf{=} - Equal 
		\item \textbf{<>} - Not Equal 
		\item \textbf{>} - Greater Than 
		\item \textbf{<} - Less Than 
		\item \textbf{>=} - Greater Than or Equal To 
		\item \textbf{<=} - Less Than or Equal To 
		\item \textbf{BETWEEN} - Between an inclusive range 
		\item \textbf{LIKE} - Search for a pattern 
		\item \textbf{IN} - To specify multiple possible values for a column 
	\end{enumerate} 
	
	\subsubsection{\textbf{Logical Operators}} 
	The following are the Logical Operators used in SQL: 
	\begin{enumerate} 
		\item \textbf{AND} - Logical AND 
		\item \textbf{OR} - Logical OR 
		\item \textbf{NOT} - Logical NOT 
	\end{enumerate} 
	
	\section{\textbf{Platform}} 
	\begin{itemize} 
		\item \textbf{Operating System:} Garuda Linux 
		\item \textbf{IDE or Text Editor:} Visual Studio Code 
	\end{itemize} 
	
	\section{\textbf{input}} 
	Given Database from the Problem Statement for the Assignment for our batch. (A1 PA15) 
	
	\section{\textbf{output}} 
	\textbf{Queries Set 1} 
	\begin{lstlisting}[language=SQL] 
		
		1. List the number of employees and average salary for employees in department 20 
		MariaDB [Company]> select avg(sal),count(*) from emp where deptno=20; 
		+-----------+----------+ 
		| avg(sal)  | count(*) | 
		+-----------+----------+ 
		| 2253.0000 |        3 | 
		+-----------+----------+ 
		1 row in set (0.001 sec) 
		
		2. List name, salary and PF amount of all employees. (PF is calculated as 10% of basic salary) 
		MariaDB [Company]> select empname, sal, sal * 0.10 as PF from emp; 
		+-----------+------+--------+ 
		| empname   | sal  | PF     | 
		+-----------+------+--------+ 
		| Harriet   | 2362 | 236.20 | 
		| Frederick | 3597 | 359.70 | 
		| Manuel    | 2773 | 277.30 | 
		| Smith     |  800 |  80.00 | 
		| Mable     | 2937 | 293.70 | 
		| Allen     | 1936 | 193.60 | 
		| Parth     | 2000 | 200.00 | 
		| Henrietta | 2780 | 278.00 | 
		| Derrick   | 2659 | 265.90 | 
		| Clarence  | 3175 | 317.50 | 
		| Effie     | 2012 | 201.20 | 
		+-----------+------+--------+ 
		11 rows in set (0.001 sec) 
		
		3. List the employee details in the ascending order of their basic salary. 
		MariaDB [Company]> select * from emp order by sal; 
		+-------+-----------+------------+------+------------+------+------+--------+ 
		| empno | empname   | job        | mgr  | hiredate   | sal  | comm | deptno | 
		+-------+-----------+------------+------+------------+------+------+--------+ 
		|  7369 | Smith     | Clerk      | 7902 | 1980-12-17 |  800 |  300 |     20 | 
		|  7499 | Allen     | Salesman   | 7698 | 1981-02-20 | 1936 |  300 |     30 | 
		|  7500 | Parth     | HR         | 7989 | 1982-03-21 | 2000 |  300 |     40 | 
		|  7941 | Effie     | Sales      | 6074 | 2071-12-03 | 2012 |  265 |     30 | 
		|  7054 | Harriet   | Research   | 7917 | 2052-12-08 | 2362 |  229 |     20 | 
		|  7705 | Derrick   | Sales      | 7234 | 2056-06-30 | 2659 |  208 |     30 | 
		|  7352 | Manuel    | Accounting | 6190 | 2084-12-21 | 2773 |  253 |     10 | 
		|  7670 | Henrietta | Sales      | 7381 | 2121-11-16 | 2780 |  217 |     30 | 
		|  7375 | Mable     | Operations | 6333 | 2097-01-17 | 2937 |  210 |     40 | 
		|  7871 | Clarence  | Operations | 6991 | 2083-03-24 | 3175 |  228 |     40 | 
		|  7128 | Frederick | Research   | 7589 | 2077-12-07 | 3597 |  203 |     20 | 
		+-------+-----------+------------+------+------------+------+------+--------+ 
		11 rows in set (0.001 sec) 
		
		4. List the employee name and hire date in the descending order of the hire date. 
		MariaDB [Company]> select empname, hiredate from emp order by hiredate desc; 
		+-----------+------------+ 
		| empname   | hiredate   | 
		+-----------+------------+ 
		| Henrietta | 2121-11-16 | 
		| Mable     | 2097-01-17 | 
		| Manuel    | 2084-12-21 | 
		| Clarence  | 2083-03-24 | 
		| Frederick | 2077-12-07 | 
		| Effie     | 2071-12-03 | 
		| Derrick   | 2056-06-30 | 
		| Harriet   | 2052-12-08 | 
		| Parth     | 1982-03-21 | 
		| Allen     | 1981-02-20 | 
		| Smith     | 1980-12-17 | 
		+-----------+------------+ 
		11 rows in set (0.001 sec) 
		
		
		5. List employee name, salary, PF, HRA, DA and gross; order the results in the ascending order of 
		gross. HRA is 50% of the salary and DA is 30% of the salary. 
		
		MariaDB [Company]> select empname, sal, sal*.10 as PF, sal*.50 as HRA, sal*.30 as DA, sal + sal*.90 as Gross from emp order by Gross; 
		+-----------+------+--------+---------+---------+---------+ 
		| empname   | sal  | PF     | HRA     | DA      | Gross   | 
		+-----------+------+--------+---------+---------+---------+ 
		| Smith     |  800 |  80.00 |  400.00 |  240.00 | 1520.00 | 
		| Allen     | 1936 | 193.60 |  968.00 |  580.80 | 3678.40 | 
		| Parth     | 2000 | 200.00 | 1000.00 |  600.00 | 3800.00 | 
		| Effie     | 2012 | 201.20 | 1006.00 |  603.60 | 3822.80 | 
		| Harriet   | 2362 | 236.20 | 1181.00 |  708.60 | 4487.80 | 
		| Derrick   | 2659 | 265.90 | 1329.50 |  797.70 | 5052.10 | 
		| Manuel    | 2773 | 277.30 | 1386.50 |  831.90 | 5268.70 | 
		| Henrietta | 2780 | 278.00 | 1390.00 |  834.00 | 5282.00 | 
		| Mable     | 2937 | 293.70 | 1468.50 |  881.10 | 5580.30 | 
		| Clarence  | 3175 | 317.50 | 1587.50 |  952.50 | 6032.50 | 
		| Frederick | 3597 | 359.70 | 1798.50 | 1079.10 | 6834.30 | 
		+-----------+------+--------+---------+---------+---------+ 
		11 rows in set (0.001 sec) 
		
		
		6. List the department numbers and number of employees in each department. 
		
		MariaDB [Company]> select deptno,count(*) from emp group by deptno; 
		+--------+----------+ 
		| deptno | count(*) | 
		+--------+----------+ 
		|     10 |        1 | 
		|     20 |        3 | 
		|     30 |        4 | 
		|     40 |        3 | 
		+--------+----------+ 
		4 rows in set (0.001 sec) 
		
		
		7. Increment the Salary of salesman by 10% of basic salary. 
		MariaDB [Company]> update emp set sal=sal+(sal*.10) where job='sales'; 
		Query OK, 3 rows affected (0.053 sec) 
		Rows matched: 3  Changed: 3  Warnings: 0 
		
		8. List the total salary, maximum and minimum salary and average salary of the employees, for 
		department 20. 
		
		MariaDB [Company]> select sum(sal),max(sal),min(sal),avg(sal) from emp where deptno=20; 
		+----------+----------+----------+-----------+ 
		| sum(sal) | max(sal) | min(sal) | avg(sal)  | 
		+----------+----------+----------+-----------+ 
		|     6759 |     3597 |      800 | 2253.0000 | 
		+----------+----------+----------+-----------+ 
		1 row in set (0.001 sec) 
		
		9. List the employees whose names contains 3 rd letter as 'I'. 
		
		MariaDB [Company]> select empname from emp where empname like '__i%'; 
		+---------+ 
		| empname | 
		+---------+ 
		| Smith   | 
		+---------+ 
		1 row in set (0.001 sec) 
		
		
		10. List the maximum salary paid to a salesman. 
		
		
		MariaDB [Company]> select *, max(sal) from emp where job='sales'; 
		+-------+-----------+-------+------+------------+------+------+--------+----------+ 
		| empno | empname   | job   | mgr  | hiredate   | sal  | comm | deptno | max(sal) | 
		+-------+-----------+-------+------+------------+------+------+--------+----------+ 
		|  7670 | Henrietta | Sales | 7381 | 2121-11-16 | 3058 |  217 |     30 |     3058 | 
		+-------+-----------+-------+------+------------+------+------+--------+----------+ 
		1 row in set (0.001 sec) 
		
		
		11. Increase the salary of salesman by 10% of their current salary. 
		
		MariaDB [Company]> update emp set sal=sal+(sal*.10) where job='sales'; 
		Query OK, 3 rows affected (0.054 sec) 
		Rows matched: 3  Changed: 3  Warnings: 0 
		
	\end{lstlisting} 
	
	\textbf{Queries Set 2:} 
	\begin{lstlisting}[language=SQL] 
		
		1. List the employee names and his annual salary dept wise. 
		
		MariaDB [Company]> select deptno, empname, sal*12 as Annual_Sal from emp order by deptno; 
		+--------+-----------+------------+ 
		| deptno | empname   | Annual_Sal | 
		+--------+-----------+------------+ 
		|     10 | Manuel    |      33276 | 
		|     20 | Harriet   |      28344 | 
		|     20 | Frederick |      43164 | 
		|     20 | Smith     |       9600 | 
		|     30 | Derrick   |      38616 | 
		|     30 | Henrietta |      40368 | 
		|     30 | Allen     |      23232 | 
		|     30 | Effie     |      29208 | 
		|     40 | Parth     |      24000 | 
		|     40 | Mable     |      35244 | 
		|     40 | Clarence  |      38100 | 
		+--------+-----------+------------+ 
		11 rows in set (0.001 sec) 
		
		2. Find out least 5 earners of the company. 
		
		MariaDB [Company]> select empname from emp order by sal asc limit 5; 
		+---------+ 
		| empname | 
		+---------+ 
		| Smith   | 
		| Allen   | 
		| Parth   | 
		| Harriet | 
		| Effie   | 
		+---------+ 
		5 rows in set (0.001 sec) 
		
		3. List the records from emp whose deptno is not in dept 
		
		MariaDB [Company]> select * from emp where deptno not in (select deptno from dept); 
		+-------+---------+------+------+------------+------+------+--------+ 
		| empno | empname | job  | mgr  | hiredate   | sal  | comm | deptno | 
		+-------+---------+------+------+------------+------+------+--------+ 
		|  7500 | Parth   | HR   | 7989 | 1982-03-21 | 2000 |  300 |     60 | 
		+-------+---------+------+------+------------+------+------+--------+ 
		1 row in set (0.001 sec) 
		
		4. List those employees whose sal is odd value. 
		
		MariaDB [Company]> select * from emp where sal % 2 != 0; 
		+-------+-----------+------------+------+------------+------+------+--------+ 
		| empno | empname   | job        | mgr  | hiredate   | sal  | comm | deptno | 
		+-------+-----------+------------+------+------------+------+------+--------+ 
		|  7128 | Frederick | Research   | 7589 | 2077-12-07 | 3597 |  203 |     20 | 
		|  7352 | Manuel    | Accounting | 6190 | 2084-12-21 | 2773 |  253 |     10 | 
		|  7375 | Mable     | Operations | 6333 | 2097-01-17 | 2937 |  210 |     40 | 
		|  7871 | Clarence  | Operations | 6991 | 2083-03-24 | 3175 |  228 |     40 | 
		+-------+-----------+------------+------+------------+------+------+--------+ 
		4 rows in set (0.001 sec) 
		
		5. List the employees whose sal contain 3 digits. 
		MariaDB [Company]> select * from emp where sal/1000 < 1; 
		+-------+---------+-------+------+------------+-----+------+--------+ 
		| empno | empname | job   | mgr  | hiredate   | sal | comm | deptno | 
		+-------+---------+-------+------+------------+-----+------+--------+ 
		|  7369 | Smith   | Clerk | 7902 | 1980-12-17 | 800 |  300 |     20 | 
		+-------+---------+-------+------+------------+-----+------+--------+ 
		1 row in set (0.001 sec) 
		
		6. List the employees who joined in the month of 'DEC' 
		MariaDB [Company]> select * from emp where hiredate like "%%%%-12-%%"; 
		+-------+-----------+------------+------+------------+------+------+--------+ 
		| empno | empname   | job        | mgr  | hiredate   | sal  | comm | deptno | 
		+-------+-----------+------------+------+------------+------+------+--------+ 
		|  7054 | Harriet   | Research   | 7917 | 2052-12-08 | 2362 |  229 |     20 | 
		|  7128 | Frederick | Research   | 7589 | 2077-12-07 | 3597 |  203 |     20 | 
		|  7352 | Manuel    | Accounting | 6190 | 2084-12-21 | 2773 |  253 |     10 | 
		|  7369 | Smith     | Clerk      | 7902 | 1980-12-17 |  800 |  300 |     20 | 
		|  7941 | Effie     | Sales      | 6074 | 2071-12-03 | 2434 |  265 |     30 | 
		+-------+-----------+------------+------+------------+------+------+--------+ 
		5 rows in set (0.001 sec) 
		
		
		7. List the employees whose names contains 'A' 
		
		MariaDB [Company]> select * from emp where empname like "A%"; 
		+-------+---------+----------+------+------------+------+------+--------+ 
		| empno | empname | job      | mgr  | hiredate   | sal  | comm | deptno | 
		+-------+---------+----------+------+------------+------+------+--------+ 
		|  7499 | Allen   | Salesman | 7698 | 1981-02-20 | 1936 |  300 |     30 | 
		+-------+---------+----------+------+------------+------+------+--------+ 
		1 row in set (0.001 sec) 
		
		8. List the maximum, minimum and average salary in the company. 
		MariaDB [Company]> select max(sal), min(sal), avg(sal) from emp; 
		+----------+----------+-----------+ 
		| max(sal) | min(sal) | avg(sal)  | 
		+----------+----------+-----------+ 
		|     3597 |      800 | 2599.6364 | 
		+----------+----------+-----------+ 
		1 row in set (0.000 sec) 
		
		9. Write a query to return the day of the week for any date(or HIRE_DATE) entered in format 
		'DD-MM-YY' 
		
		MariaDB [Company]> select dayname(hiredate) from emp; 
		+-------------------+ 
		| dayname(hiredate) | 
		+-------------------+ 
		| Sunday            | 
		| Tuesday           | 
		| Thursday          | 
		| Wednesday         | 
		| Thursday          | 
		| Friday            | 
		| Sunday            | 
		| Sunday            | 
		| Friday            | 
		| Wednesday         | 
		| Thursday          | 
		+-------------------+ 
		11 rows in set (0.001 sec) 
		
		10. Count the no of characters in employee name without considering spaces for each name. 
		
		MariaDB [Company]> select empname, length(replace(empname, ' ', '')) + 1 as length from emp; 
		+-----------+--------+ 
		| empname   | length | 
		+-----------+--------+ 
		| Harriet   |      8 | 
		| Frederick |     10 | 
		| Manuel    |      7 | 
		| Smith     |      6 | 
		| Mable     |      6 | 
		| Allen     |      6 | 
		| Parth     |      6 | 
		| Henrietta |     10 | 
		| Derrick   |      8 | 
		| Clarence  |      9 | 
		| Effie     |      6 | 
		+-----------+--------+ 
		11 rows in set (0.001 sec) 
		
		11. List the employees who are drawing less than 1000. sort the output by salary. 
		
		MariaDB [Company]> select * from emp where sal < 1000 order by sal; 
		+-------+---------+-------+------+------------+-----+------+--------+ 
		| empno | empname | job   | mgr  | hiredate   | sal | comm | deptno | 
		+-------+---------+-------+------+------------+-----+------+--------+ 
		|  7369 | Smith   | Clerk | 7902 | 1980-12-17 | 800 |  300 |     20 | 
		+-------+---------+-------+------+------------+-----+------+--------+ 
		1 row in set (0.002 sec) 
	\end{lstlisting} 
	\textbf{Queries Set 3:} 
	\begin{lstlisting}[language=SQL] 
		1. Write a query in SQL to display the unique designations for the employees. 
		
		MariaDB [Company]> select distinct job from emp; 
		+------------+ 
		| job        | 
		+------------+ 
		| Research   | 
		| Accounting | 
		| Clerk      | 
		| Operations | 
		| Salesman   | 
		| HR         | 
		| Sales      | 
		+------------+ 
		7 rows in set (0.001 sec) 
		
		2. Delete Employees who joined in Year 1980. 
		MariaDB [Company]> delete from emp where year(hiredate) = 1980; 
		Query OK, 1 row affected (0.127 sec) 
		
		3. Increase the salary of Managers by 20% of their current salary. 
		MariaDB [Company]> update emp set sal = sal + sal*0.2 where job = 'Manager'; 
		Query OK, 0 rows affected (0.001 sec) 
		Rows matched: 0  Changed: 0  Warnings: 0 
		
		4. List employees not belonging to department 30, 40, or 10. 
		
		MariaDB [Company]> select * from emp where deptno not in (30, 40, 10); 
		+-------+-----------+----------+------+------------+------+------+--------+ 
		| empno | empname   | job      | mgr  | hiredate   | sal  | comm | deptno | 
		+-------+-----------+----------+------+------------+------+------+--------+ 
		|  7054 | Harriet   | Research | 7917 | 2052-12-08 | 2362 |  229 |     20 | 
		|  7128 | Frederick | Research | 7589 | 2077-12-07 | 3597 |  203 |     20 | 
		|  7500 | Parth     | HR       | 7989 | 1982-03-21 | 2000 |  300 |     60 | 
		+-------+-----------+----------+------+------------+------+------+--------+ 
		3 rows in set (0.001 sec) 
		
		5. List the different designations in the company. 
		
		MariaDB [Company]> select distinct job from emp; 
		+------------+ 
		| job        | 
		+------------+ 
		| Research   | 
		| Accounting | 
		| Operations | 
		| Salesman   | 
		| HR         | 
		| Sales      | 
		+------------+ 
		6 rows in set (0.001 sec) 
		
		6. List the names of employees who are not eligible for commission. 
		
		MariaDB [Company]> select * from emp where sal < 1000; 
		Empty set (0.001 sec) 
		
		7. List employees whose names either start or end with "S". 
		
		MariaDB [Company]>   select * from emp where empname like 'S%' or empname like '%S'; 
		Empty set (0.001 sec) 
		
		8. List employees whose names have letter "A" as second letter in their names. 
		
		MariaDB [Company]> select * from emp where empname like '_A%'; 
		+-------+---------+------------+------+------------+------+------+--------+ 
		| empno | empname | job        | mgr  | hiredate   | sal  | comm | deptno | 
		+-------+---------+------------+------+------------+------+------+--------+ 
		|  7054 | Harriet | Research   | 7917 | 2052-12-08 | 2362 |  229 |     20 | 
		|  7352 | Manuel  | Accounting | 6190 | 2084-12-21 | 2773 |  253 |     10 | 
		|  7375 | Mable   | Operations | 6333 | 2097-01-17 | 2937 |  210 |     40 | 
		|  7500 | Parth   | HR         | 7989 | 1982-03-21 | 2000 |  300 |     60 | 
		+-------+---------+------------+------+------------+------+------+--------+ 
		4 rows in set (0.001 sec) 
		
		9. List the number of employees working with the company. 
		
		MariaDB [Company]> select count(*) from emp; 
		+----------+ 
		| count(*) | 
		+----------+ 
		|       10 | 
		+----------+ 
		1 row in set (0.001 sec) 
		
		10. List the emps with hiredate in format June 4,1988. 
		
		MariaDB [Company]> select * from emp where hiredate = '1988-06-04'; 
		Empty set (0.001 sec) 
		
		11. List the salesmen who get the commission within a range of 200 and 5000. 
		
		MariaDB [Company]> select * from emp where job = 'Sales' and comm between 200 and 5000; 
		+-------+-----------+-------+------+------------+------+------+--------+ 
		| empno | empname   | job   | mgr  | hiredate   | sal  | comm | deptno | 
		+-------+-----------+-------+------+------------+------+------+--------+ 
		|  7670 | Henrietta | Sales | 7381 | 2121-11-16 | 3364 |  217 |     30 | 
		|  7705 | Derrick   | Sales | 7234 | 2056-06-30 | 3218 |  208 |     30 | 
		|  7941 | Effie     | Sales | 6074 | 2071-12-03 | 2434 |  265 |     30 | 
		+-------+-----------+-------+------+------------+------+------+--------+ 
		3 rows in set (0.001 sec) 
		
	\end{lstlisting} 
	
	\textbf{Queries Set 4:} 
	\begin{lstlisting}[language=SQL] 
		## Set 4 
		
		1. List names of employees who are more than 2 years old in the company. 
		
		MariaDB [Company]> select empname from emp where datediff(curdate(), hiredate)/365 > 2; 
		+---------+ 
		| empname | 
		+---------+ 
		| Allen   | 
		| Parth   | 
		+---------+ 
		2 rows in set (0.002 sec) 
		
		2. List the employee details in the ascending order of their basic salary. 
		
		MariaDB [Company]> select * from emp order by sal; 
		+-------+-----------+------------+------+------------+------+------+--------+ 
		| empno | empname   | job        | mgr  | hiredate   | sal  | comm | deptno | 
		+-------+-----------+------------+------+------------+------+------+--------+ 
		|  7499 | Allen     | Salesman   | 7698 | 1981-02-20 | 1936 |  300 |     30 | 
		|  7500 | Parth     | HR         | 7989 | 1982-03-21 | 2000 |  300 |     60 | 
		|  7054 | Harriet   | Research   | 7917 | 2052-12-08 | 2362 |  229 |     20 | 
		|  7941 | Effie     | Sales      | 6074 | 2071-12-03 | 2434 |  265 |     30 | 
		|  7352 | Manuel    | Accounting | 6190 | 2084-12-21 | 2773 |  253 |     10 | 
		|  7375 | Mable     | Operations | 6333 | 2097-01-17 | 2937 |  210 |     40 | 
		|  7871 | Clarence  | Operations | 6991 | 2083-03-24 | 3175 |  228 |     40 | 
		|  7705 | Derrick   | Sales      | 7234 | 2056-06-30 | 3218 |  208 |     30 | 
		|  7670 | Henrietta | Sales      | 7381 | 2121-11-16 | 3364 |  217 |     30 | 
		|  7128 | Frederick | Research   | 7589 | 2077-12-07 | 3597 |  203 |     20 | 
		+-------+-----------+------------+------+------------+------+------+--------+ 
		10 rows in set (0.001 sec) 
		
		3. Display the employees who have more salary as that of Smith 
		
		MariaDB [Company]> select * from emp where sal > (select sal from emp where empname = 'Manuel'); 
		+-------+-----------+------------+------+------------+------+------+--------+ 
		| empno | empname   | job        | mgr  | hiredate   | sal  | comm | deptno | 
		+-------+-----------+------------+------+------------+------+------+--------+ 
		|  7128 | Frederick | Research   | 7589 | 2077-12-07 | 3597 |  203 |     20 | 
		|  7375 | Mable     | Operations | 6333 | 2097-01-17 | 2937 |  210 |     40 | 
		|  7670 | Henrietta | Sales      | 7381 | 2121-11-16 | 3364 |  217 |     30 | 
		|  7705 | Derrick   | Sales      | 7234 | 2056-06-30 | 3218 |  208 |     30 | 
		|  7871 | Clarence  | Operations | 6991 | 2083-03-24 | 3175 |  228 |     40 | 
		+-------+-----------+------------+------+------------+------+------+--------+ 
		5 rows in set (0.001 sec) 
		
		4. Increment the salary of Emp no. 9180 by 10% of his current salary. 
		
		MariaDB [Company]> select * from emp where empno = 9180; 
		+-------+---------+-----------+------+------------+------+------+--------+ 
		| empno | empname | job       | mgr  | hiredate   | sal  | comm | deptno | 
		+-------+---------+-----------+------+------------+------+------+--------+ 
		|  9180 | Jesse   | Acounting | 6534 | 2101-08-22 | 3562 | 1420 |     10 | 
		+-------+---------+-----------+------+------------+------+------+--------+ 
		1 row in set (0.001 sec) 
		
		5. List the employees whose salary is between 10000 and 25000. 
		
		MariaDB [Company]> select * from emp where sal between 10000 and 25000; 
		+-------+---------+------+------+------------+-------+------+--------+ 
		| empno | empname | job  | mgr  | hiredate   | sal   | comm | deptno | 
		+-------+---------+------+------+------------+-------+------+--------+ 
		|  7500 | Parth   | HR   | 7989 | 1982-03-21 | 11100 |  300 |     60 | 
		+-------+---------+------+------+------------+-------+------+--------+ 
		1 row in set (0.001 sec) 
		
		6. List the names of employees who are not eligible for commission. 
		
		MariaDB [Company]> select * from emp where sal < 1000; 
		+-------+---------+-------+------+------------+-----+------+--------+ 
		| empno | empname | job   | mgr  | hiredate   | sal | comm | deptno | 
		+-------+---------+-------+------+------------+-----+------+--------+ 
		|  7941 | Effie   | Sales | 6074 | 2071-12-03 | 700 |  265 |     30 | 
		+-------+---------+-------+------+------------+-----+------+--------+ 
		1 row in set (0.001 sec) 
		
		7. Increment the Salary of Research by 10% of basic salary. 
		
		MariaDB [Company]> update emp set sal = sal + sal*0.1 where job = "Research"; 
		Query OK, 2 rows affected (0.054 sec) 
		Rows matched: 2  Changed: 2  Warnings: 0 
		
		8. List the total salary, maximum and minimum salary and average salary of the employees jobwise. 
		
		MariaDB [Company]> select job, sum(sal) as total, max(sal) as max, min(sal) as min, avg(sal) as avg from emp group by job; 
		+------------+-------+-------+-------+------------+ 
		| job        | total | max   | min   | avg        | 
		+------------+-------+-------+-------+------------+ 
		| Accounting |  2773 |  2773 |  2773 |  2773.0000 | 
		| Acounting  |  3562 |  3562 |  3562 |  3562.0000 | 
		| HR         | 11100 | 11100 | 11100 | 11100.0000 | 
		| Operations |  6112 |  3175 |  2937 |  3056.0000 | 
		| Research   |  6555 |  3957 |  2598 |  3277.5000 | 
		| Sales      |  7282 |  3364 |   700 |  2427.3333 | 
		| Salesman   |  1936 |  1936 |  1936 |  1936.0000 | 
		+------------+-------+-------+-------+------------+ 
		7 rows in set (0.001 sec) 
		
		9. Delete the Employee whose name starts with P. 
		
		MariaDB [Company]> delete from emp where empname like 'P%'; 
		Query OK, 1 row affected (0.106 sec) 
		
		10. List the employees whose designation is "Research" and commission is > 500. 
		+-------+-----------+----------+------+------------+------+------+--------+ 
		| empno | empname   | job      | mgr  | hiredate   | sal  | comm | deptno | 
		+-------+-----------+----------+------+------------+------+------+--------+ 
		|  7054 | Harriet   | Research | 7917 | 2052-12-08 | 2598 |  610 |     20 | 
		|  7128 | Frederick | Research | 7589 | 2077-12-07 | 3957 |  610 |     20 | 
		+-------+-----------+----------+------+------------+------+------+--------+ 
		2 rows in set (0.001 sec) 
		
		11. List employees belonging to department 20, 30, 40. 
		
		MariaDB [Company]> select * from emp where deptno in (20, 30, 40); 
		+-------+-----------+------------+------+------------+------+------+--------+ 
		| empno | empname   | job        | mgr  | hiredate   | sal  | comm | deptno | 
		+-------+-----------+------------+------+------------+------+------+--------+ 
		|  7054 | Harriet   | Research   | 7917 | 2052-12-08 | 2598 |  610 |     20 | 
		|  7128 | Frederick | Research   | 7589 | 2077-12-07 | 3957 |  610 |     20 | 
		|  7375 | Mable     | Operations | 6333 | 2097-01-17 | 2937 |  210 |     40 | 
		|  7499 | Allen     | Salesman   | 7698 | 1981-02-20 | 1936 |  300 |     30 | 
		|  7670 | Henrietta | Sales      | 7381 | 2121-11-16 | 3364 |  217 |     30 | 
		|  7705 | Derrick   | Sales      | 7234 | 2056-06-30 | 3218 |  208 |     30 | 
		|  7871 | Clarence  | Operations | 6991 | 2083-03-24 | 3175 |  228 |     40 | 
		|  7941 | Effie     | Sales      | 6074 | 2071-12-03 |  700 |  265 |     30 | 
		+-------+-----------+------------+------+------------+------+------+--------+ 
		8 rows in set (0.001 sec) 
		
	\end{lstlisting} 
	\textbf{Queries Set 5:} 
	\begin{lstlisting}[language=SQL] 
		1. List the employee names and his annual salary Job wise. 
		
		MariaDB [Company]> select job, empname, sal*12 as annual from emp; 
		+------------+-----------+--------+ 
		| job        | empname   | annual | 
		+------------+-----------+--------+ 
		| Research   | Harriet   |  31176 | 
		| Research   | Frederick |  47484 | 
		| Accounting | Manuel    |  33276 | 
		| Operations | Mable     |  35244 | 
		| Salesman   | Allen     |  23232 | 
		| Sales      | Henrietta |  40368 | 
		| Sales      | Derrick   |  38616 | 
		| Operations | Clarence  |  38100 | 
		| Sales      | Effie     |   8400 | 
		| Acounting  | Jesse     |  42744 | 
		+------------+-----------+--------+ 
		10 rows in set (0.001 sec) 
		
		2. Delete the Employee whose name starts with A & R 
		
		delete from emp where empname like 'A%' or empname like 'R%'; 
		Query OK, 1 row affected (0.163 sec) 
		
		3. Increment the salary of Emp no. 7000 by 30% of his current salary. 
		
		MariaDB [Company]> update emp set sal = sal + sal*0.3 where empno = 7000; 
		Query OK, 1 row affected (0.043 sec) 
		Rows matched: 1  Changed: 1  Warnings: 0 
		
		4. List the total salary, maximum and minimum salary and average salary of the employees hire date wise. 
		
		MariaDB [Company]> select hiredate, sum(sal) as total, max(sal) as max, min(sal) as min, avg(sal) as avg from emp group by hiredate; 
		+------------+-------+------+------+-----------+ 
		| hiredate   | total | max  | min  | avg       | 
		+------------+-------+------+------+-----------+ 
		| 2052-12-08 |  2598 | 2598 | 2598 | 2598.0000 | 
		| 2056-06-30 |  3218 | 3218 | 3218 | 3218.0000 | 
		| 2071-12-03 |   700 |  700 |  700 |  700.0000 | 
		| 2077-12-07 |  3957 | 3957 | 3957 | 3957.0000 | 
		| 2083-03-24 |  3175 | 3175 | 3175 | 3175.0000 | 
		| 2084-12-21 |  2773 | 2773 | 2773 | 2773.0000 | 
		| 2097-01-17 |  2937 | 2937 | 2937 | 2937.0000 | 
		| 2101-08-22 | 12662 | 9100 | 3562 | 6331.0000 | 
		| 2121-11-16 |  3364 | 3364 | 3364 | 3364.0000 | 
		+------------+-------+------+------+-----------+ 
		9 rows in set (0.001 sec) 
		
		5. List the employees whose names contains last letter as 'T'. 
		
		MariaDB [Company]>  select * from emp where empname like '%T';  
		+-------+---------+----------+------+------------+------+------+--------+ 
		| empno | empname | job      | mgr  | hiredate   | sal  | comm | deptno | 
		+-------+---------+----------+------+------------+------+------+--------+ 
		|  7054 | Harriet | Research | 7917 | 2052-12-08 | 2598 |  610 |     20 | 
		+-------+---------+----------+------+------------+------+------+--------+ 
		1 row in set (0.001 sec) 
		
		6. Display the employees who have less salary as that of Ankush 
		
		MariaDB [Company]> select * from emp where sal < (select sal from emp where empname = 'Ankush'); 
		Empty set (0.001 sec) 
		
		7. Display the employees who have salary between 10000 
		
		MariaDB [Company]> select * from emp where sal between 10000 and 20000; 
		Empty set (0.001 sec) 
		
		8. List employees belonging to department 30, 40, or 10. 
		
		MariaDB [Company]> select * from emp where deptno in (30, 40, 10); 
		+-------+-----------+------------+------+------------+------+------+--------+ 
		| empno | empname   | job        | mgr  | hiredate   | sal  | comm | deptno | 
		+-------+-----------+------------+------+------------+------+------+--------+ 
		|  7352 | Manuel    | Accounting | 6190 | 2084-12-21 | 2773 |  253 |     10 | 
		|  7375 | Mable     | Operations | 6333 | 2097-01-17 | 2937 |  210 |     40 | 
		|  7670 | Henrietta | Sales      | 7381 | 2121-11-16 | 3364 |  217 |     30 | 
		|  7705 | Derrick   | Sales      | 7234 | 2056-06-30 | 3218 |  208 |     30 | 
		|  7871 | Clarence  | Operations | 6991 | 2083-03-24 | 3175 |  228 |     40 | 
		|  7941 | Effie     | Sales      | 6074 | 2071-12-03 |  700 |  265 |     30 | 
		|  9180 | Jesse     | Acounting  | 6534 | 2101-08-22 | 3562 | 1420 |     10 | 
		+-------+-----------+------------+------+------------+------+------+--------+ 
		7 rows in set (0.001 sec) 
		
		9. List the employees whose designation is 'Research' and sal is > 5000. 
		
		MariaDB [Company]>  select * from emp where job = 'Research' and sal > 5000; 
		Empty set (0.001 sec) 
		
		10. List the employees details descending wise whose designation is 'Research' and commission is > 500. 
		
		MariaDB [Company]> select * from emp where job = 'Research' and comm > 500 order by comm desc; 
		+-------+-----------+----------+------+------------+------+------+--------+ 
		| empno | empname   | job      | mgr  | hiredate   | sal  | comm | deptno | 
		+-------+-----------+----------+------+------------+------+------+--------+ 
		|  7054 | Harriet   | Research | 7917 | 2052-12-08 | 2598 |  610 |     20 | 
		|  7128 | Frederick | Research | 7589 | 2077-12-07 | 3957 |  610 |     20 | 
		+-------+-----------+----------+------+------------+------+------+--------+ 
		2 rows in set (0.001 sec) 
		
	\end{lstlisting} 
	
	\section{\textbf{Conclusion}} 
	Thus, we have learned SQL DML commands, SELECT Command with SQL operators thoroughly 
	
	\section{\textbf{FAQs}} 
	\subsection{\textbf{What is the between Truncate table and Drop table?}} 
	\begin{itemize} 
		\item Truncate table is used to delete all the records from the table. It is faster than 
		\item Drop table is used to delete the table completely from the database. 
		\item Truncate table is faster than drop table. 
		\item Truncate table command cannot be rolled back. 
		\item Drop table command can be rolled back. 
	\end{itemize} 
	\textbf{Example:} 
	\begin{itemize} 
		\item Truncate table Command; 
		\begin{verbatim} 
			Truncate table emp; 
		\end{verbatim} 
		\item Drop table Command; 
		\begin{verbatim} 
			Drop table emp; 
		\end{verbatim} 
	\end{itemize} 
	
	\subsection{\textbf{How is the pattern matching done in SQL?}} 
	\begin{itemize} 
		\item The pattern matching is done using the LIKE operator. 
		\item The LIKE operator is used in the WHERE clause of the SELECT statement. 
		\item The LIKE operator is used to search for a specific pattern in a column. 
		\item The pattern can be a string or a number. 
		\item The pattern can contain special characters. 
		\item The special characters are: 
		\begin{itemize} 
			\item \% - The percent sign represents zero, one, or multiple characters. 
			\item \_ - The underscore represents a single character. 
		\end{itemize} 
	\end{itemize} 
	\textbf{The Syntax of the command is:} 
	\begin{verbatim} 
		SELECT column_name(s) 
		FROM table_name 
		WHERE column_name LIKE pattern; 
	\end{verbatim} 
	\textbf{Example:} 
	\begin{verbatim} 
		SELECT * FROM emp WHERE empname LIKE 'S%'; 
		SELECT * FROM emp WHERE empname LIKE 'S__'; 
	\end{verbatim} 
	
	\subsection{\textbf{Write a DELETE command to delete all the records from CUSTOMER table}} 
	\begin{verbatim} 
		DELETE FROM CUSTOMER; 
	\end{verbatim} 
\end{document}